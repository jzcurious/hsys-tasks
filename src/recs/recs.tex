\documentclass[12pt,a4paper]{article}

% Поддержка русского языка
\usepackage[T2A]{fontenc}
\usepackage[utf8]{inputenc}
\usepackage[russian]{babel}

% Геометрия страницы
\usepackage[left=2.5cm,right=2.5cm,top=2.5cm,bottom=2.5cm]{geometry}

% Шрифты
\usepackage{paratype} % PT Sans, PT Serif
\renewcommand{\familydefault}{\sfdefault} % Sans Serif по умолчанию

% Цвета
\usepackage{xcolor}
\definecolor{sectioncolor}{RGB}{0,102,204}
\definecolor{subsectioncolor}{RGB}{51,51,51}
\definecolor{codebackground}{RGB}{245,245,245}
\definecolor{linkcolor}{RGB}{0,102,204}

% Гиперссылки
\usepackage{hyperref}
\hypersetup{
    colorlinks=true,
    linkcolor=linkcolor,
    urlcolor=linkcolor,
    citecolor=linkcolor,
    pdftitle={Рекомендации по проектированию университетского курса на уровне production-quality},
    pdfauthor={},
    pdfsubject={Методические рекомендации},
    pdfkeywords={образование, программирование, курс, университет}
}

% Таблицы
\usepackage{booktabs}
\usepackage{longtable}
\usepackage{array}
\usepackage{multirow}
\newcolumntype{L}[1]{>{\raggedright\arraybackslash}p{#1}}
\newcolumntype{C}[1]{>{\centering\arraybackslash}p{#1}}

% Списки
\usepackage{enumitem}
\setlist{nosep}

% Код
\usepackage{listings}
\lstset{
    basicstyle=\ttfamily\small,
    backgroundcolor=\color{codebackground},
    frame=single,
    breaklines=true,
    columns=flexible
}

% Форматирование заголовков
\usepackage{titlesec}
\titleformat{\section}
    {\normalfont\Large\bfseries\color{sectioncolor}}
    {\thesection}{1em}{}
\titleformat{\subsection}
    {\normalfont\large\bfseries\color{subsectioncolor}}
    {\thesubsection}{1em}{}
\titleformat{\subsubsection}
    {\normalfont\normalsize\bfseries}
    {\thesubsubsection}{1em}{}

% Spacing
\usepackage{parskip}
\setlength{\parindent}{0pt}
\setlength{\parskip}{6pt}

% Графика
\usepackage{graphicx}

% Колонтитулы
\usepackage{fancyhdr}
\pagestyle{fancy}
\fancyhf{}
\fancyhead[L]{\small\textit{Рекомендации по проектированию университетского курса}}
\fancyhead[R]{\small\thepage}
\renewcommand{\headrulewidth}{0.4pt}

% Checkmark и X для таблиц
\usepackage{pifont}
\newcommand{\cmark}{\ding{51}}
\newcommand{\xmark}{\ding{55}}

\begin{document}

% Титульная страница
\begin{titlepage}
    \centering
    \vspace*{3cm}

    {\Huge\bfseries Рекомендации по проектированию университетского курса на уровне production-quality\par}

    \vspace{2cm}

    {\Large Методические рекомендации для преподавателей технических дисциплин в области разработки программного обеспечения\par}

    \vfill

    {\large\textit{СФУ, ИКИТ, каф. ВпВ\\
    ст. преп. С.\,А.\,Тарасов\\
    к.т.н, доцент, зав. каф., Д.\,А.\,Кузьмин\\}\par}

    \vspace{1cm}

    {\large 2025\par}
\end{titlepage}

% Содержание
\tableofcontents
\newpage

\section*{Преамбула}
\addcontentsline{toc}{section}{Преамбула}

Данные рекомендации адресованы преподавателям технических дисциплин (программирование, HPC, системное ПО, ML и т.\,д.), которые хотят вывести свои курсы на уровень, соответствующий современным требованиям индустрии и ведущих международных образовательных программ.

Рекомендации извлечены из анализа конкретного курса, доказавшего свою эффективность (курс «Гибридные вычислительные системы», СФУ, ИКИТ, каф. ВпВ, ст. преп. С.\,А.\,Тарасов), и обобщены для применения к широкому спектру технических дисциплин. Подходы, описанные в документе, перекликаются с принципами построения ряда известных курсов ведущих университетов, таких как CMU 15-213 (Computer Systems), Stanford CS 149 (Parallel Computing), MIT 6.172 (Performance Engineering of Software Systems).

\subsection*{Как пользоваться этим документом}

Переход от «классического» курса к описанной модели~--- масштабная задача. Не обязательно внедрять всё сразу. Рекомендуемая дорожная карта:

\begin{itemize}
    \item \textbf{Год 1 (фундамент):} внедрите Git, систему сборки и автоматическое тестирование. Сформулируйте пререквизиты и критерии оценивания.
    \item \textbf{Год 2 (сквозной проект):} перепроектируйте лабораторные как последовательность работ, формирующих единый продукт. Добавьте прикладной контекст.
    \item \textbf{Год 3 (полный цикл):} подключите CI/CD, бенчмаркинг, код-ревью, паттерны проектирования в контексте предметной области.
\end{itemize}

Каждый раздел самодостаточен: можно внедрять рекомендации по отдельности, получая пользу на каждом шаге.

\newpage

\section{Сквозной проект вместо изолированных лабораторных}

\subsection{Проблема}

Типичная структура практики: 10~лабораторных, каждая~--- самостоятельная задача, не связанная с предыдущими. Студент пишет «одноразовый» код, который никогда не используется повторно. Навыки не кумулятивны.

\subsection{Рекомендация}

Проектируйте практическую часть как \textbf{последовательность работ, формирующих единый программный продукт}. Каждая лабораторная должна порождать модуль (библиотеку, компонент), который используется в последующих работах.

\subsection{Что это даёт}

\begin{itemize}
    \item \textbf{Кумулятивность знаний:} каждая следующая работа опирается на предыдущую, студент не может «перепрыгнуть», не освоив фундамент.
    \item \textbf{Реалистичность:} в индустрии никто не пишет изолированные программы~--- всегда есть кодовая база, в которую нужно интегрироваться.
    \item \textbf{Мотивация:} студент видит, как его работа складывается в целостную систему, а не пропадает после сдачи.
    \item \textbf{Качество кода:} если библиотека из работы №3 используется в работе №7, студент вынужден писать чистый, тестируемый, документированный код~--- иначе он сам пострадает позже.
\end{itemize}

\subsection{Пример реализации}

В курсе-образце 10~лабораторных последовательно создают библиотеку для реализации моделей глубокого обучения на CUDA: от базовых абстракций (тензоры, аллокаторы) до слоёв нейронной сети и интеграции с PyTorch.

\subsection{Антипаттерн}

10~не связанных между собой задач: «напишите сортировку пузырьком», «реализуйте стек», «сделайте калькулятор». Каждая сдаётся и забывается. К концу курса у студента нет ни продукта, ни системного понимания.

\section{Полный инженерный цикл в каждой работе}

\subsection{Проблема}

В большинстве курсов «сдать лабораторную» = «показать, что программа выдаёт правильный ответ». Тестирование, измерение производительности и анализ результатов игнорируются.

\subsection{Рекомендация}

Каждая лабораторная работа должна включать следующие этапы:

\begin{enumerate}
    \item \textbf{Проектирование и реализация} (обязательно)~--- разработка классов, функций, модулей
    \item \textbf{Интеграция} (обязательно)~--- встраивание нового кода в существующую кодовую базу
    \item \textbf{Тестирование} (обязательно)~--- написание автоматических тестов (unit tests, integration tests)
    \item \textbf{Бенчмаркинг} (желательно)~--- измерение производительности с использованием профессиональных инструментов
    \item \textbf{Анализ} (обязательно)~--- интерпретация результатов, выявление узких мест, формулирование выводов
\end{enumerate}

Таким образом, обязательных этапов четыре; бенчмаркинг включается тогда, когда это целесообразно для конкретной работы (например, при оптимизации вычислительных ядер), и является обязательным хотя бы для 2–3 работ в курсе.

\subsection{Что это даёт}

\begin{itemize}
    \item Формирует \textbf{инженерную культуру}, а не привычку «лишь бы заработало».
    \item Студент учится не только писать код, но и \textbf{доказывать его корректность} (тесты) и \textbf{измерять его качество} (бенчмарки).
    \item Этап анализа развивает \textbf{критическое мышление}~--- студент не просто получает числа, а объясняет их.
\end{itemize}

\subsection{Инструменты}

\begin{table}[h]
\centering
\begin{tabular}{@{}lp{10cm}@{}}
\toprule
\textbf{Назначение} & \textbf{Примеры} \\
\midrule
Тестирование & Google Test, Catch2 (C++); PyTest (Python); JUnit (Java); \texttt{cargo test} (Rust) \\
Бенчмаркинг & Google Benchmark (C++); pytest-benchmark (Python); JMH (Java); criterion (Rust/Haskell) \\
Профилирование & perf, Valgrind, VTune, Nsight (GPU), gprof, py-spy и др. \\
\bottomrule
\end{tabular}
\end{table}

\subsection{Антипаттерн}

Преподаватель запускает программу студента, смотрит на вывод, сравнивает с эталоном «на глаз». Нет тестов, нет метрик, нет анализа. Студент не знает, работает ли его код на граничных случаях.

\section{Современный технологический стек}

\subsection{Проблема}

Многие курсы используют устаревшие инструменты: ручная компиляция через \texttt{gcc} в терминале без системы сборки, отсутствие VCS, IDE, отсутствие линтеров и форматтеров. Студент выходит с навыками, неприменимыми в индустрии.

\subsection{Рекомендация}

Технологический стек курса должен \textbf{максимально соответствовать тому, что используется в индустрии и открытых проектах}. Минимальный стандарт:

\begin{longtable}{@{}L{4cm}L{6cm}L{4.5cm}@{}}
\toprule
\textbf{Категория} & \textbf{Минимальный стандарт} & \textbf{Комментарий} \\
\midrule
\endfirsthead
\toprule
\textbf{Категория} & \textbf{Минимальный стандарт} & \textbf{Комментарий} \\
\midrule
\endhead
\bottomrule
\endfoot

Система сборки & CMake (C/C++), Gradle (Java), Cargo (Rust), pyproject.toml + setuptools / Poetry (Python) & Никаких ручных \texttt{gcc main.c -o main} \\

Контроль версий & Git + GitHub/GitLab & Обязательно с первой работы \\

Тестирование & Фреймворк тестирования, соответствующий языку & Не «проверка глазами», а автоматические тесты \\

Статический анализ / линтинг & clang-tidy (C++), ruff/pylint (Python), ESLint (JS), clippy (Rust) & Формирует привычку к чистому коду \\

Форматирование & clang-format, black/ruff format, prettier, rustfmt & Единый стиль, нет споров об отступах \\

LSP / IDE & clangd, pyright, rust-analyzer + любой редактор & Навигация по коду, автодополнение, диагностика \\

Документация & README.md в каждом проекте, docstrings/doxygen & Умение описать, что делает код \\

\end{longtable}

\subsection{Что это даёт}

\begin{itemize}
    \item Студент выходит из курса с \textbf{навыками, которые сразу применимы} на стажировке или в open source.
    \item Снижается \textbf{когнитивный разрыв} между «университетским кодом» и «реальным кодом».
    \item Преподаватель получает \textbf{автоматизированную проверку} (CI) вместо ручного запуска студенческих программ.
\end{itemize}

\subsection{Антипаттерн}

Студент компилирует код командой \texttt{gcc -o lab3 lab3.c}, запускает, копирует вывод в Word-файл, отправляет в eКурсы. Нет истории изменений, нет воспроизводимости, нет автоматизации.

\section{Код-ревью и практика коллаборации}

\subsection{Проблема}

Студент пишет код, сдаёт его преподавателю и получает оценку. Никто, кроме автора и (иногда) преподавателя, этот код не видит. Студент не учится читать чужой код, давать конструктивную обратную связь и принимать критику своего кода. При этом в индустрии code review~--- один из ключевых процессов обеспечения качества.

\subsection{Рекомендация}

Внедрите практику \textbf{код-ревью} как обязательный элемент сдачи работ:

\begin{itemize}
    \item \textbf{Pull Request как форма сдачи:} студент создаёт PR в репозитории курса, преподаватель (или TA) оставляет комментарии, студент исправляет замечания. Работа принимается после approve.
    \item \textbf{Peer code review:} студенты проверяют код друг друга (по 1–2 рецензента на PR). Это формирует навык чтения чужого кода и умение формулировать замечания конструктивно.
    \item \textbf{Осмысленные коммиты:} введите требования к commit messages (например, формат Conventional Commits или просто правило «коммит должен объяснять \textit{зачем}, а не только \textit{что}»).
\end{itemize}

\subsection{Что это даёт}

\begin{itemize}
    \item Формирует навык, который студент будет использовать \textbf{с первого дня на работе}.
    \item Улучшает качество кода: зная, что код прочитает другой человек, студент пишет аккуратнее.
    \item Развивает \textbf{коммуникативные навыки} в техническом контексте.
    \item Создаёт \textbf{коллективную ответственность} за качество.
\end{itemize}

\subsection{Практические аспекты}

\begin{itemize}
    \item При большом потоке используйте ротацию рецензентов (каждый студент рецензирует 1–2 работы, а не все).
    \item Предоставьте чек-лист для рецензента: на что обращать внимание (читаемость, тесты, именование, граничные случаи).
    \item Первые 2–3 работы~--- ревью только от преподавателя/TA, чтобы задать стандарт. Далее~--- подключение peer review.
\end{itemize}

\section{Паттерны проектирования в контексте предметной области}

\subsection{Проблема}

Паттерны проектирования (Design Patterns) часто преподаются в отрыве от предметной области~--- абстрактные примеры с «фабриками уток» и «стратегиями сортировки», которые студент забывает через неделю.

\subsection{Рекомендация}

Вводите паттерны проектирования \textbf{в контексте конкретных задач курса}, показывая, какую проблему они решают. Важен принцип: студент сталкивается с проблемой в своём проекте, и паттерн предлагается как решение, а не наоборот.

\subsection{Примеры из курса-образца (C++/CUDA)}

\begin{longtable}{@{}L{3.5cm}L{5.5cm}L{5.5cm}@{}}
\toprule
\textbf{Паттерн} & \textbf{Как используется} & \textbf{Какую проблему решает} \\
\midrule
\endfirsthead
\toprule
\textbf{Паттерн} & \textbf{Как используется} & \textbf{Какую проблему решает} \\
\midrule
\endhead
\bottomrule
\endfoot

RAII & Управление GPU-ресурсами (cudaMalloc/cudaFree) & Предотвращение утечек памяти GPU \\

Data + View & Разделение владения данными и представления (тензор vs slice тензора) & Безопасная работа с памятью без лишнего копирования \\

Expression Templates & Ленивые вычисления в линейной алгебре & Избежание создания временных объектов, оптимизация производительности \\

CRTP & Статический полиморфизм для слоёв нейронной сети & Полиморфизм без виртуальных вызовов \\

Policy/Strategy & Выбор стратегии аллокации (CPU/GPU) & Абстракция от конкретного типа памяти \\

\end{longtable}

\subsection{Примеры для других языков и доменов}

\begin{longtable}{@{}L{3cm}L{3cm}L{8.5cm}@{}}
\toprule
\textbf{Язык/домен} & \textbf{Паттерн} & \textbf{Контекст} \\
\midrule
\endfirsthead
\toprule
\textbf{Язык/домен} & \textbf{Паттерн} & \textbf{Контекст} \\
\midrule
\endhead
\bottomrule
\endfoot

Python (ML) & Strategy & Выбор оптимизатора (SGD, Adam) через единый интерфейс \\

Python (ML) & Factory & Создание моделей по конфигурации (YAML → объект модели) \\

Java (веб) & Observer & Система событий в веб-фреймворке \\

Rust (системное ПО) & Builder & Конфигурация сетевого сервера с множеством параметров \\

Rust (системное ПО) & Newtype & Типобезопасные обёртки для ID, портов, путей \\

\end{longtable}

\subsection{Что это даёт}

\begin{itemize}
    \item Студент понимает \textbf{зачем} нужен паттерн, а не просто запоминает его UML-диаграмму.
    \item Паттерны усваиваются через практику, а не через лекцию.
\end{itemize}

\section{Открытость и прозрачность материалов}

\subsection{Проблема}

Задания курса существуют в виде PDF-файлов на eКурсах с ограниченным доступом. Критерии оценивания размыты или отсутствуют. Студент не знает, чего от него ждут. Материалы невоспроизводимы.

\subsection{Рекомендация}

\begin{itemize}
    \item \textbf{Опубликуйте задания на GitHub/GitLab}~--- в открытом доступе или в приватном репозитории с доступом для студентов.
    \item \textbf{Приложите к каждой работе явные критерии оценивания}~--- рубрики с баллами за каждый компонент (корректность, тесты, производительность, качество кода, документация).
    \item \textbf{Предоставьте шаблон проекта} (template repository) с настроенной системой сборки, CI и структурой каталогов~--- чтобы студент тратил время на содержание, а не на настройку окружения.
\end{itemize}

\subsection{Что это даёт}

\begin{itemize}
    \item \textbf{Прозрачность:} студент понимает, за что получает (или не получает) баллы.
    \item \textbf{Воспроизводимость:} курс может быть передан другому преподавателю, адаптирован, улучшен.
    \item \textbf{Портфолио:} студент формирует GitHub-профиль с реальными проектами~--- это работает лучше, чем строчка в дипломе.
    \item \textbf{Качество через открытость:} когда задания открыты, их видят коллеги, индустрия, другие студенты~--- это мотивирует поддерживать высокий уровень.
\end{itemize}

\subsection{Академическая честность}

Открытость заданий создаёт риск плагиата (код предыдущих студентов доступен, задания можно решить с помощью LLM-сервисов). Стратегии управления этим риском:

\begin{itemize}
    \item \textbf{Культурный подход:} объясняйте студентам, что цель~--- научиться, а не «сдать».
    \item \textbf{Сквозной проект:} скопировать одну работу легко, но интегрировать чужой код в свой проект, не понимая его,~--- сложно.
    \item \textbf{Ежегодная ротация параметров заданий:} меняются конкретные задачи, архитектура, численные параметры при сохранении общей структуры курса.
    \item \textbf{Устная защита:} студент должен объяснить свой код и ответить на вопросы по нему. Это одновременно и проверка, и обучающий элемент.
    \item \textbf{Автоматическое обнаружение плагиата:} инструменты MOSS, JPlag, Copydetect позволяют сравнивать решения между студентами и с открытыми источниками.
\end{itemize}

\subsection{Антипаттерн}

Задание в закрытом доступе на eКурсах, критерии «на усмотрение преподавателя», разные требования для разных студентов, никакой обратной связи кроме оценки в ведомости.

\section{Прикладной контекст}

\subsection{Проблема}

Многие курсы преподают технологии «в вакууме»: CUDA-курс, где пишут сложение векторов; MPI-курс, где считают число~$\pi$; курс по ОС, где нет связи с реальными системами.

\subsection{Рекомендация}

Выберите \textbf{прикладную тему, которая проходит через весь курс} и мотивирует изучение каждой технической темы. Эта тема должна быть:

\begin{itemize}
    \item \textbf{Актуальной}~--- чтобы студент понимал, зачем ему это в карьере
    \item \textbf{Достаточно сложной}~--- чтобы покрыть все темы курса
    \item \textbf{Измеримой}~--- чтобы можно было оценить результат (производительность, точность)
\end{itemize}

\subsection{Примеры}

\begin{longtable}{@{}L{3.5cm}L{3.5cm}L{7.5cm}@{}}
\toprule
\textbf{Курс} & \textbf{Прикладная тема} & \textbf{Как работает} \\
\midrule
\endfirsthead
\toprule
\textbf{Курс} & \textbf{Прикладная тема} & \textbf{Как работает} \\
\midrule
\endhead
\bottomrule
\endfoot

GPU Computing & Глубокое обучение & Каждая тема CUDA мотивирована конкретной потребностью DL: матричные операции → shared memory, mixed precision → Tensor Cores, автодифф → CUDA Graphs \\

Параллельное программирование (MPI) & Вычислительная гидродинамика (CFD) & Метод решёточных уравнений Больцмана~--- каждая тема MPI мотивирована: декомпозиция домена → MPI\_Cart, обмен гало → MPI\_Sendrecv, сбор результатов → MPI\_Gather \\

Операционные системы & Контейнеризация / «Напиши свой Docker» & Namespaces, cgroups, chroot, overlayfs~--- каждая тема ОС мотивирована конкретным аспектом контейнера \\

Компиляторы & Язык для научных вычислений & Лексер, парсер, AST, кодогенерация~--- каждый этап мотивирован конкретной фичей языка \\

\end{longtable}

\subsection{Что это даёт}

\begin{itemize}
    \item \textbf{Мотивация:} студент видит, зачем ему очередная техническая тема.
    \item \textbf{Целостность:} знания не фрагментированы, а образуют систему.
    \item \textbf{Результат:} к концу курса у студента есть не набор упражнений, а \textbf{работающий продукт}.
\end{itemize}

\subsection{Антипаттерн}

Десять лабораторных: сложение векторов, транспонирование матрицы, «Hello World» из нескольких потоков, вычисление числа~$\pi$ методом Монте-Карло... К концу курса студент не может объяснить, для чего всё это нужно в реальном проекте.

\section{Явные пререквизиты и управление входным порогом}

\subsection{Проблема}

Курс предполагает определённые знания (например, C++, линейная алгебра, Linux), но это нигде не зафиксировано. Студенты с недостаточной подготовкой проваливаются на 2–3 неделе, демотивируя и себя, и преподавателя.

\subsection{Рекомендация}

\begin{itemize}
    \item \textbf{Явно сформулируйте пререквизиты} в описании курса: что студент должен знать и уметь до начала.
    \item \textbf{Предоставьте материалы для самопроверки}~--- диагностический тест или список задач: «если вы не можете решить эти 5~задач, вам нужно повторить X перед курсом».
    \item \textbf{Если пререквизиты не обеспечены учебным планом}~--- добавьте boot camp (1–2~недели интенсива в начале) или предоставьте материалы для самостоятельного изучения.
\end{itemize}

\subsection{Пример}

Для курса по GPU Computing пререквизиты могут быть:

\begin{itemize}
    \item \textbf{C++:} указатели, работа с памятью (new/delete), шаблоны, basic STL, RAII~--- «если вы не понимаете, чем \texttt{unique\_ptr} отличается от raw pointer, пройдите модуль X»
    \item \textbf{Linux:} командная строка, SSH
    \item \textbf{Математика:} линейная алгебра (матричные операции), мат.~анализ (дифференцирование)
\end{itemize}

\subsection{Что это даёт}

\begin{itemize}
    \item \textbf{Честный входной порог:} студент осознанно принимает решение о записи на курс.
    \item \textbf{Снижение отсева:} меньше студентов «тонет» на первых неделях.
    \item \textbf{Экономия времени преподавателя:} не нужно тратить аудиторные часы на повторение базового материала.
\end{itemize}

\subsection{Антипаттерн}

В описании курса: «Предварительные требования: знание языка программирования». Какого языка? На каком уровне? Что именно должен уметь студент? Неясно.

\section{Контакт с реальным оборудованием и инфраструктурой}

\subsection{Проблема}

Студенты решают задачи «на бумаге» или в симуляторах. Они никогда не видели реальный сервер, не подключались к удалённому кластеру, не запускали задачу через job scheduler. При выходе на работу обнаруживается, что навыки работы с реальной инфраструктурой отсутствуют полностью.

\subsection{Рекомендация}

Обеспечьте студентам доступ к \textbf{реальному оборудованию}, соответствующему предмету курса:

\begin{longtable}{@{}L{3.5cm}L{4cm}L{7cm}@{}}
\toprule
\textbf{Курс} & \textbf{Оборудование} & \textbf{Минимальная конфигурация} \\
\midrule
\endfirsthead
\toprule
\textbf{Курс} & \textbf{Оборудование} & \textbf{Минимальная конфигурация} \\
\midrule
\endhead
\bottomrule
\endfoot

GPU Computing & Сервер с NVIDIA GPU & 1~сервер с GPU (Tesla T4 / A100) или облако \\

HPC / MPI & Кластер с job scheduler & 4~узла с SLURM~--- достаточно для учебных задач \\

Сети & Сетевая лаборатория & Управляемые коммутаторы, маршрутизаторы \\

Встроенные системы & Отладочные платы & STM32, Raspberry Pi, FPGA-платы \\

\end{longtable}

\subsection{Как получить ресурсы}

\begin{itemize}
    \item \textbf{Образовательные программы вендоров:} NVIDIA DLI (бесплатный доступ к GPU), AWS Academy, Google for Education, JetBrains Educational~--- предоставляют кредиты и инфраструктуру для ВУЗов.
    \item \textbf{Cloud-fallback:} Google Colab, Kaggle Notebooks~--- бесплатный доступ к GPU для учебных задач.
    \item \textbf{Собственная инфраструктура кафедры/факультета:} даже один сервер с GPU или мини-кластер из 4~узлов~--- лучше, чем ничего. Стоит рассмотреть совместное использование с научными группами.
\end{itemize}

\subsection{Практические аспекты}

\begin{itemize}
    \item Подготовьте \textbf{инструкцию по доступу} к инфраструктуре: SSH-ключи, VPN, SLURM-скрипты, правила использования.
    \item Назначьте \textbf{ответственного за администрирование} (ассистент, лаборант, сисадмин кафедры)~--- без этого инфраструктура быстро деградирует.
    \item Установите \textbf{квоты и правила fair use}~--- чтобы один студент не монополизировал GPU накануне дедлайна.
\end{itemize}

\subsection{Что это даёт}

\begin{itemize}
    \item \textbf{Реалистичный опыт:} студент учится работать с инфраструктурой, а не только с локальным компьютером.
    \item \textbf{Навык удалённой работы:} SSH, tmux/screen, job schedulers~--- навыки, востребованные в индустрии и науке.
    \item \textbf{Масштаб задач:} на реальном оборудовании можно запускать задачи, которые невозможны на ноутбуке.
\end{itemize}

\subsection{Антипаттерн}

Курс по параллельному программированию, где все задачи выполняются на двухъядерном ноутбуке студента. «Параллелизм» сводится к \texttt{\#pragma omp parallel for} на массиве из 1000~элементов.

\section{Масштабируемость и автоматизация}

\subsection{Проблема}

Описанные выше практики (код-ревью, индивидуальная обратная связь, бенчмаркинг) требуют значительных трудозатрат преподавателя. Но при группе в 15--20~студентов это вполне реализуемо с использованием средств автоматизации.

\subsection{Рекомендация}

\textbf{Автоматизируйте всё, что можно автоматизировать:}

\begin{itemize}
    \item CI/CD на каждый PR: автоматический запуск тестов, линтера, форматтера. Если тесты не проходят~--- PR не принимается. Это снимает с преподавателя 80\% рутинной проверки.
    \item Autograding: GitHub Classroom, Gradescope, или собственные скрипты, запускающие тестовые сценарии и генерирующие отчёт.
    \item Автоматическая проверка стиля кода: clang-format / black / rustfmt в CI~--- исключает споры о форматировании.
\end{itemize}

\textbf{Привлекайте ассистентов:}

\begin{itemize}
    \item Ассистенты из числа студентов проводят код-ревью, помогают на лабораторных, отвечают на вопросы в чате.
    \item Это полезно и для самих ассистентов: преподавание углубляет понимание материала.
\end{itemize}

\textbf{Используйте асинхронные каналы коммуникации:}

\begin{itemize}
    \item Max-чат курса для оперативных вопросов.
    \item Discussions на GitHub для вопросов по заданиям (ответы видны всем, не нужно отвечать одно и то же 30~раз).
\end{itemize}

\subsection{Что это даёт}

\begin{itemize}
    \item Курс остаётся \textbf{качественным при масштабировании}~--- 20--30~студентов получают такую же обратную связь, как 5.
    \item Преподаватель фокусируется на \textbf{содержательных вещах} (дизайн заданий, лекции, сложные вопросы), а не на рутинной проверке.
    \item Студенты получают \textbf{быструю обратную связь} (CI за минуты, а не проверка через 2~недели).
\end{itemize}

\section{Регулярное обновление содержания}

\subsection{Проблема}

Курс написан один раз и не обновляется годами. Технологии устаревают, инструменты исчезают, а студенты продолжают изучать устаревшие API.

\subsection{Рекомендация}

\begin{itemize}
    \item \textbf{Пересматривайте содержание ежегодно:} добавляйте новые архитектуры, API, инструменты.
    \item \textbf{Ведите changelog курса}~--- как у софтверных проектов. Каждый семестр фиксируйте, что изменилось: новые задания, обновлённые инструменты, исправленные ошибки. Используйте git tags для версионирования (например, \texttt{v2024-fall}, \texttt{v2025-spring}).
    \item \textbf{Следите за индустрией:} конференции (SC, GTC, CppCon, PyCon), блоги (NVIDIA Developer Blog, Intel Developer Zone), release notes ключевых инструментов.
    \item \textbf{Привлекайте обратную связь:} анонимные опросы студентов в конце курса, отзывы выпускников через 1–2~года после окончания.
    \item \textbf{Правило обновления:} если $\geq$30\% API или инструментов, используемых в курсе, помечены как deprecated~--- это сигнал к срочному обновлению.
\end{itemize}

\subsection{Что это даёт}

\begin{itemize}
    \item \textbf{Актуальность:} студенты изучают то, что используется в индустрии \textit{сейчас}, а не 5~лет назад.
    \item \textbf{Доверие:} студенты (и работодатели) ценят курсы, которые живут и развиваются.
    \item \textbf{Эволюция, а не революция:} ежегодные небольшие обновления проще, чем полная переработка раз в 5~лет.
\end{itemize}

\subsection{Антипаттерн}

Курс 2025~года, в котором используется Python~2, CUDA Toolkit~9.0, и задания набраны в Microsoft Word~2003. На вопрос «а почему не X?» преподаватель отвечает: «Это проверено временем».

\section{Командная работа и коммуникация}

\subsection{Проблема}

Все лабораторные выполняются индивидуально. Студент никогда не работал в команде над общей кодовой базой, не решал merge-конфликты, не распределял задачи, не координировал работу с другими разработчиками. При этом в индустрии индивидуальная разработка~--- исключение, а не правило.

\subsection{Рекомендация}

Включите в курс \textbf{хотя бы один командный проект} (финальный проект или 2–3~последние лабораторные):

\begin{itemize}
    \item \textbf{Команды по 2–3~человека}~--- достаточно, чтобы создать потребность в координации, но не настолько много, чтобы кто-то мог «спрятаться».
    \item \textbf{Работа в общем репозитории:} branching strategy, merge/rebase, разрешение конфликтов.
    \item \textbf{Распределение ответственности:} каждый член команды отвечает за свой модуль, но все модули должны работать вместе.
    \item \textbf{Peer evaluation:} участники команды анонимно оценивают вклад друг друга. Это помогает обнаружить и скорректировать дисбаланс.
\end{itemize}

\subsection{Что это даёт}

\begin{itemize}
    \item \textbf{Навык командной разработки}~--- один из самых востребованных в индустрии.
    \item \textbf{Практика Git-workflow} (feature branches, pull requests, code review) в реалистичном контексте.
    \item Опыт \textbf{технической коммуникации:} описание интерфейсов, обсуждение архитектурных решений, написание документации для коллег.
\end{itemize}

\section{Оценка эффективности курса}

\subsection{Проблема}

Преподаватель не знает, работают ли его нововведения. Единственный «метрика»~--- процент сдавших экзамен, который мало что говорит о реальном качестве обучения.

\subsection{Рекомендация}

Введите систему оценки эффективности курса по нескольким измерениям:

\begin{longtable}{@{}L{4cm}L{5cm}L{5.5cm}@{}}
\toprule
\textbf{Метрика} & \textbf{Как измерять} & \textbf{Когда} \\
\midrule
\endfirsthead
\toprule
\textbf{Метрика} & \textbf{Как измерять} & \textbf{Когда} \\
\midrule
\endhead
\bottomrule
\endfoot

Удовлетворённость студентов & Анонимный опрос (Google Forms) с открытыми и закрытыми вопросами & В конце каждого семестра \\

Retention rate & Доля студентов, завершивших курс (сдавших $\geq$80\% работ) & По итогам семестра \\

Качество кода & Средний балл по рубрике, количество замечаний на код-ревью, покрытие тестами & В процессе курса \\

Карьерные результаты & Опрос выпускников через 1–2~года: стажировки, трудоустройство, использование навыков & Ежегодно \\

Внешняя валидация & Звёзды / форки на GitHub, отзывы коллег, приглашения на конференции & Постоянно \\

\end{longtable}

\subsection{Что это даёт}

\begin{itemize}
    \item \textbf{Обоснование:} данные для аргументации перед администрацией (зачем нужно оборудование, время на обновление курса).
    \item \textbf{Итеративное улучшение:} выявление слабых мест и целенаправленная доработка.
    \item \textbf{Мотивация преподавателя:} видимый результат своей работы.
\end{itemize}

\section{Чек-лист самопроверки для преподавателя}

Перед запуском курса или его очередной итерации пройдите по этому списку. Пункты разделены на три категории по приоритету.

\subsection{Критически важно}

\begin{longtable}{@{}cL{10cm}cc@{}}
\toprule
\textbf{\#} & \textbf{Вопрос} & \textbf{Раздел} & \textbf{\cmark/\xmark} \\
\midrule
\endfirsthead
\toprule
\textbf{\#} & \textbf{Вопрос} & \textbf{Раздел} & \textbf{\cmark/\xmark} \\
\midrule
\endhead
\bottomrule
\endfoot

1 & Есть ли у курса сквозной проект, объединяющий лабораторные? & §1 & \\
2 & Используется ли Git с первой работы? & §3 & \\
3 & Есть ли автоматизированная система сборки (CMake, Gradle, Cargo, ...)? & §3 & \\
4 & Включает ли каждая работа автоматическое тестирование? & §2 & \\
5 & Опубликованы ли задания с явными критериями оценивания? & §6 & \\
6 & Сформулированы ли пререквизиты явно? & §8 & \\

\end{longtable}

\subsection{Важно}

\begin{longtable}{@{}cL{10cm}cc@{}}
\toprule
\textbf{\#} & \textbf{Вопрос} & \textbf{Раздел} & \textbf{\cmark/\xmark} \\
\midrule
\endfirsthead
\toprule
\textbf{\#} & \textbf{Вопрос} & \textbf{Раздел} & \textbf{\cmark/\xmark} \\
\midrule
\endhead
\bottomrule
\endfoot

7 & Есть ли прикладная тема, мотивирующая каждый технический модуль? & §7 & \\
8 & Внедрено ли код-ревью (преподавательское или peer)? & §4 & \\
9 & Соответствует ли технологический стек индустриальным стандартам? & §3 & \\
10 & Есть ли доступ к реальному оборудованию (или облачный fallback)? & §9 & \\
11 & Включены ли паттерны проектирования в контексте предметной области? & §5 & \\
12 & Настроена ли автоматизация (CI/CD, autograding)? & §10 & \\

\end{longtable}

\subsection{Желательно}

\begin{longtable}{@{}cL{10cm}cc@{}}
\toprule
\textbf{\#} & \textbf{Вопрос} & \textbf{Раздел} & \textbf{\cmark/\xmark} \\
\midrule
\endfirsthead
\toprule
\textbf{\#} & \textbf{Вопрос} & \textbf{Раздел} & \textbf{\cmark/\xmark} \\
\midrule
\endhead
\bottomrule
\endfoot

13 & Включает ли курс бенчмаркинг в $\geq$2 работах? & §2 & \\
14 & Есть ли командный проект? & §12 & \\
15 & Выходит ли студент с портфолио (GitHub), а не только с оценкой? & §6 & \\
16 & Обновлялось ли содержание курса в последние 12~месяцев? & §11 & \\
17 & Проводится ли оценка эффективности курса (опросы, метрики)? & §13 & \\
18 & Есть ли стратегия обеспечения академической честности? & §6 & \\

\end{longtable}

\section*{Заключение}
\addcontentsline{toc}{section}{Заключение}

Перечисленные рекомендации не претендуют на полноту~--- каждый курс уникален, и конкретные решения зависят от предмета, аудитории, ресурсов и культуры ВУЗа. Однако принципы, лежащие в основе рекомендаций, универсальны:

\begin{itemize}
    \item \textbf{Практика должна быть кумулятивной}, а не одноразовой.
    \item \textbf{Инструменты должны быть индустриальными}, а не «учебными».
    \item \textbf{Критерии должны быть прозрачными}, а не подразумеваемыми.
    \item \textbf{Обратная связь должна быть быстрой}, а не запоздалой.
    \item \textbf{Курс должен жить и развиваться}, а не застывать.
\end{itemize}

Главный вопрос, который стоит задавать себе: \textit{«Если бы я нанимал выпускника этого курса~--- был бы я доволен его подготовкой?»}

\end{document}
