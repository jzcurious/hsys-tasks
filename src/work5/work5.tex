\documentclass[12pt]{article}

\usepackage[T2A]{fontenc}
\usepackage[utf8]{inputenc}
\usepackage[english, russian]{babel}
\usepackage[a4paper,top=2cm,bottom=2cm,left=3cm,right=3cm,marginparwidth=1.75cm]{geometry}
\usepackage[style=russian]{csquotes}
\usepackage{microtype}
\usepackage{hyperref}
\usepackage{caption}
\usepackage{graphicx}

\emergencystretch=2em

\title{Практическая работа №5}
\author{ст. преп. каф. ВпВ ИКИТ СФУ Тарасов С. А.}
\date{}

\begin{document}
\maketitle

\section*{Цель работы}
Закрепить навык работы с разделяемой памятью и примитивами синхронизации нитей. Освоить программирование уровня warp'а и технику широковещания регистров.

\section*{Задание}
\begin{enumerate}
    \item Разработать кёрнел \texttt{kernel\_vecred\_nobr} (1), который принимает экземпляр \texttt{VectorView} по значению и вычисляет сумму элементов вектора. Для хранения частичных сумм блоков использовать разделяемую память.

    \item Разработать кёрнел \texttt{kernel\_vecred\_br} (2), который принимает экземпляр \texttt{VectorView} по значению и вычисляет сумму элементов вектора, используя функцию \texttt{\_\_shfl\_down\_sync}. Для хранения частичных сумм warp'ов использовать разделяемую память.

    \item Используя фреймворк \texttt{Google Test}, разработать модульные тесты для функций (1) и (2) с размерами векторов $n \in \{1, 2, 3, 127, 129, 512, 541, 1037\}$. В качестве эталона для сравнения использовать результат аналогичной операции для \texttt{Eigen::Matrix} (\texttt{sum}); для верификации результатов применять макрос \texttt{EXPECT\_NEAR} с абсолютной точностью $10^{-4}$.

    \item Используя фреймворк \texttt{Google Benchmark}, разработать бенчмарки  для функций (1) и (2) с размерами векторов $n \in \{8 \cdot 2^0, 8 \cdot 2^1, 8 \cdot 2^2, ..., 8 \cdot 2^{28}\}$. Бенчмарки должны игнорировать время, затраченное на выделение, копирование и освобождение памяти. Для корректного измерения времени выполнения CUDA-кода необходимо использовать \texttt{CUDA Events API}.

    \item Построить графики реальной вычислительной сложности для (1) и (2).

    \item Построить график ускорения (2) относительно (1).

    \item Объяснить экспериментальные результаты.

    \item Подготовить отчёт, содержащий:
    \begin{itemize}
        \item ключевые фрагменты кода;
        \item ссылку на репозиторий с полной реализацией;
        \item графики результатов измерений;
        \item анализ и интерпретацию полученных результатов.
    \end{itemize}
\end{enumerate}

\section*{Критерии оценки}
\begin{itemize}
    \item \textbf{Корректность реализации и тестирование (50\%):}
    \begin{itemize}
        \item отсутствие утечек памяти, корректная работа с \texttt{CUDA API};
        \item правильность результатов редукции векторов различных размеров;
        \item полнота тестового покрытия, включая граничные случаи;
        \item соответствие результатов эталонной реализации.
    \end{itemize}

    \item \textbf{Качество кода и архитектура (25\%):}
    \begin{itemize}
        \item чистота архитектуры, разделение ответственности между классами;
        \item единообразие стиля, качество форматирования и читаемость кода.
    \end{itemize}

    \item \textbf{Качество вычислительного эксперимента (15\%):}
    \begin{itemize}
        \item корректность методики измерений производительности;
        \item глубина анализа результатов, сравнение с теоретическими оценками.
    \end{itemize}

    \item \textbf{Документация и оформление (10\%):}
    \begin{itemize}
        \item полнота и структурированность отчёта;
        \item ясность изложения;
        \item оформление репозитория;
        \item оформление отчёта (\href{https://sfu.ru/sapi/file-upload/325396c75b669b84763c10b7b5c73ec1.pdf}{СТУ 7.5–07–2021}).
    \end{itemize}
\end{itemize}

\section*{Рекомендации по выполнению}
\begin{itemize}
    \item Реализуйте эффективный алгоритм параллельной редукции, рассмотренный на лекции.
    \item Ознакомьтесь с \href{https://docs.nvidia.com/cuda/cuda-c-programming-guide/#warp-shuffle-functions}{официальным руководством CUDA}.
\end{itemize}

\end{document}
