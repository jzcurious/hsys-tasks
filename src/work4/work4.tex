\documentclass[12pt]{article}

\usepackage[T2A]{fontenc}
\usepackage[utf8]{inputenc}
\usepackage[english, russian]{babel}
\usepackage[a4paper,top=2cm,bottom=2cm,left=3cm,right=3cm,marginparwidth=1.75cm]{geometry}
\usepackage[style=russian]{csquotes}
\usepackage{microtype}
\usepackage{hyperref}
\usepackage{caption}
\usepackage{graphicx}

\emergencystretch=2em

\title{Практическая работа №4}
\author{ст. преп. каф. ВпВ ИКИТ СФУ Тарасов С. А.}
\date{}

\begin{document}
\maketitle

\section*{Цель работы}
Получить навыки программирования тензорных ядер CUDA.

\section*{Задание}
\begin{enumerate}
    \item Разработать кёрнел \texttt{kernel\_matmul\_wmma}, который принимает объекты \texttt{MatrixView} по значению и вычисляет произведение матриц, используя \texttt{CUDA WMMA} (см. рис.~\ref{fig:mmblock}).

    \item Перегрузить оператор \texttt{operator*} для класса \texttt{Matrix}, используя указанный кёрнел.

    \item Используя фреймворк \texttt{Google Test}, разработать модульные тесты для \texttt{operator*} со следующими размерами матриц: $A$ ($m \times k$) и $B$ ($k \times n$), где $m, n, k \in \{16, 32, 64, 128, 256, 512\}$. В качестве эталона для сравнения использовать результат аналогичной операции для \texttt{Eigen::Matrix<Eigen::half, Eigen::Dynamic, Eigen::Dynamic, Eigen::RowMajor>}; для верификации результатов применять метод \texttt{Eigen::MatrixXf::isApprox} с абсолютной точностью $10^{-2}$.

    \item Используя фреймворк \texttt{Google Benchmark}, разработать бенчмарки для \texttt{operator*} со следующими размерами матриц: $n \times n$, где $n \in \{16, 32, 64, 128,$ $256, 512, 1024\}$. Бенчмарки должны игнорировать время, затраченное на выделение, копирование и освобождение памяти. Для корректного измерения времени выполнения CUDA-кода необходимо использовать \texttt{CUDA Events API}.

    \item Построить график реальной вычислительной сложности умножения матриц типа \texttt{Matrix} с помощью новой реализации \texttt{operator*}, а также аналогичный график для предыдущей версии \texttt{operator*}.

    \item Построить график ускорения (\texttt{speedup}) новой реализации \texttt{operator*} относительно предыдущей версии.

    \item Объяснить экспериментальные результаты.

    \item Подготовить отчёт, содержащий:
    \begin{itemize}
        \item ключевые фрагменты кода;
        \item ссылку на репозиторий с полной реализацией;
        \item графики результатов измерений;
        \item анализ и интерпретацию полученных результатов.
    \end{itemize}
\end{enumerate}

\section*{Критерии оценки}
\begin{itemize}
    \item \textbf{Корректность реализации и тестирование (50\%):}
    \begin{itemize}
        \item отсутствие утечек памяти, корректная работа с \texttt{CUDA API};
        \item правильность результатов умножения матриц различных размеров;
        \item полнота тестового покрытия, включая граничные случаи;
        \item соответствие результатов эталонной реализации.
    \end{itemize}

    \item \textbf{Качество кода и архитектура (25\%):}
    \begin{itemize}
        \item чистота архитектуры, разделение ответственности между классами;
        \item единообразие стиля, качество форматирования и читаемость кода.
    \end{itemize}

    \item \textbf{Качество вычислительного эксперимента (15\%):}
    \begin{itemize}
        \item корректность методики измерений производительности;
        \item глубина анализа результатов, сравнение с теоретическими оценками.
    \end{itemize}

    \item \textbf{Документация и оформление (10\%):}
    \begin{itemize}
        \item полнота и структурированность отчёта;
        \item ясность изложения;
        \item оформление репозитория;
        \item оформление отчёта (\href{https://sfu.ru/sapi/file-upload/325396c75b669b84763c10b7b5c73ec1.pdf}{СТУ 7.5–07–2021}).
    \end{itemize}
\end{itemize}

\section*{Рекомендации по выполнению}
\begin{itemize}
    \item Используйте паттерн проектирования \texttt{стратегия} при разработке \texttt{operator*}.
    \item Ознакомьтесь с \href{https://docs.nvidia.com/cuda/cuda-c-programming-guide/#warp-matrix-functions}{официальным руководством CUDA}.
\end{itemize}

\begin{figure}[ht!]
    \includegraphics[width=1\textwidth]{wmma.png}
    \caption{Схема произведения матриц с помощью WMMA}
    \label{fig:mmblock}
\end{figure}

\end{document}
