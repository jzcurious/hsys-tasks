\documentclass[12pt]{article}

\usepackage[T2A]{fontenc}
\usepackage[utf8]{inputenc}
\usepackage[english, russian]{babel}
\usepackage[a4paper,top=2cm,bottom=2cm,left=3cm,right=3cm,marginparwidth=1.75cm]{geometry}
\usepackage[style=russian]{csquotes}
\usepackage{microtype}
\usepackage{hyperref}
\usepackage{caption}
\usepackage{graphicx}
\usepackage{seqsplit}
% \usepackage{breqn}
\usepackage{amsmath}
\usepackage{booktabs}

\usepackage{xcolor}
\usepackage{listings}
\lstset{
    language=C++,                 % язык программирования
    basicstyle=\ttfamily\small,   % шрифт
    keywordstyle=\color{blue},    % цвет ключевых слов
    stringstyle=\color{red},      % цвет строк
    commentstyle=\color{green},   % цвет комментариев
    numbers=left,                 % нумерация строк слева
    numberstyle=\tiny,            % размер цифр
    stepnumber=1,                 % шаг нумерации
    frame=single,                 % рамка вокруг кода
    breaklines=true,              % перенос длинных строк
    tabsize=1                     % размер табуляции
}
\renewcommand{\lstlistingname}{Листинг}

\emergencystretch=2em

\title{Практическая работа №6}
\author{ст. преп. каф. ВпВ ИКИТ СФУ Тарасов С. А.}
\date{}

\begin{document}
\maketitle

\section*{Цель работы}
Сформировать навыки разработки \texttt{CUDA}-ядер, совместимых с \texttt{PyTorch}.

\section*{Задание}
\begin{enumerate}
    \item Разработать кёрнел \texttt{hsys::nn::functional::kernels::conv2d} (см. листинг \ref{lst:conv2d}),
    который вычисляет пакетную многоканальную двумерную свертку (кросс-корреляцию):
    \begin{multline}
    (\mathbf{X} \star \mathbf{W})[n, c_{out}, h_{out}, w_{out}] = \\
    \sum_{c_{in}=0}^{C_{in}-1} \sum_{r=0}^{k-1} \sum_{q=0}^{k-1} \hat{\mathbf{X}}[n, c_{in}, h_{out} \cdot s+r-p, w_{out} \cdot s+q-p] \cdot \mathbf{W}[c_{out}, c_{in}, r, q]
    \end{multline}
    \begin{multline}
    \hat{\mathbf{X}}[n, c_{out}, h, w] =
        \begin{cases}
        \mathbf{X}[n, c_{out}, h, w],  & 0 \leq h < H_{in}~and~0 \leq w < W_{in} \\
        0,~otherwise
        \end{cases}
    \end{multline}
    Где
    \begin{itemize}
    \item $\mathbf{X}$ --- массив размера $[N, C_{in}, H_{in}, W_{in}]$ (пакет входных изображений),
    \item $\mathbf{W}$ --- массив размера $[C_{out}, C_{in}, k, k]$ (набор ядер/весов свертки),
    \item $N$ --- размер пакета (batch size),
    \item $C_{in}$ --- количество каналов входного изображения,
    \item $H_{in}$ --- высота входного изображения,
    \item $W_{in}$ --- ширина входного изображения,
    \item $C_{out}$ --- количество каналов выходного изображения,
    \item $k$ --- размер квадратного фильтра,
    \item $s$ --- шаг свертки (stride),
    \item $p$ --- отступ (padding).
    \end{itemize}

    \begin{lstlisting}[
        caption={Декларация conv2d},       % 1. подпись
        label={lst:conv2d},                % 1. метка для \ref{}
        float=h,                           % позиционирование как у figure
    ]
    template <class T>
    using view_4d_t = torch::PackedTensorAccessor32<
        T, 4, torch::RestrictPtrTraits
    >;

    template <
        class AtomT,
        long filter_size,
        long stride = 1,
        long padding = 0,
        long block_size = 16
    >
    __global__ void conv2d(
        view_4d_t<AtomT> output,
        const view_4d_t<AtomT> input,
        const view_4d_t<AtomT> weights
    );
    \end{lstlisting}

    \item Используя фреймворк \texttt{Google Test}, разработать модульные тесты для \texttt{hsys::nn::functional::kernels::conv2d}. Тестовые сценарии должны покрывать параметры, приведённые в табл. \ref{tab:parameters}. Корректность вычислений проверять путем сравнения с эталонной реализацией \texttt{torch::nn::functional::conv2d}, используя функцию \texttt{torch::allclose} с допусками \texttt{atol} = \texttt{rtol} = $10^{-5}$.

    \begin{table}[h!]
    \centering
    \caption{Параметры для тестирования (декартово произведение)}
    \label{tab:parameters}

    \vspace{0.5cm}

    % First sub-table
    \begin{tabular}{ccc}
    \toprule
    k & s & p \\
    \midrule
    3 & 2 & 1 \\
    3 & 1 & 1 \\
    7 & 2 & 3 \\
    1 & 2 & 0 \\
    \bottomrule
    \end{tabular}

    \vspace{0.5cm}

    {\Large$\times$}

    \vspace{0.5cm}

    % Second sub-table
    \begin{tabular}{cccc}
    \toprule
    C_{in} & H_{in} & W_{in} & C_{out} \\
    \midrule
    3   & 244 & 244 & 64  \\
    64  & 112 & 112 & 64  \\
    64  & 56  & 56  & 64  \\
    128 & 28  & 28  & 128 \\
    64  & 56  & 56  & 128 \\
    128 & 28  & 28  & 256 \\
    256 & 14  & 14  & 512 \\
    256 & 14  & 14  & 256 \\
    512 & 7   & 7   & 512 \\
    \bottomrule
    \end{tabular}

    \end{table}

    \item Подготовить отчёт, содержащий:
    \begin{itemize}
        \item ключевые фрагменты кода;
        \item ссылку на репозиторий с полной реализацией.
    \end{itemize}
\end{enumerate}

\section*{Критерии оценки}
\begin{itemize}
    \item \textbf{Корректность реализации и тестирование (65\%):}
    \begin{itemize}
        \item корректная работа с \texttt{CUDA API} и \texttt{PyTorch C++ API};
        \item полнота тестового покрытия, включая граничные случаи;
        \item соответствие результатов эталонной реализации.
    \end{itemize}

    \item \textbf{Качество кода и архитектура (25\%):}
    \begin{itemize}
        \item чистота архитектуры;
        \item единообразие стиля, качество форматирования и читаемость кода.
    \end{itemize}

    \item \textbf{Документация и оформление (10\%):}
    \begin{itemize}
        \item полнота и структурированность отчёта;
        \item ясность изложения;
        \item оформление репозитория;
        \item оформление отчёта (\href{https://sfu.ru/sapi/file-upload/325396c75b669b84763c10b7b5c73ec1.pdf}{СТУ 7.5–07–2021}).
    \end{itemize}
\end{itemize}

\section*{Рекомендации по выполнению}
\begin{itemize}
    \item Реализуйте эффективный алгоритм параллельной свертки (tiling, shared memory), рассмотренный на лекции.
    \item Обратите внимание, что параметры \texttt{filter\_size}, \texttt{stride}, \texttt{padding} передаются как шаблонные аргументы (compile-time constants), что позволяет компилятору развернуть циклы (loop unrolling).
    \item Полезные ссылки:
    \begin{itemize}
        \item \href{https://docs.pytorch.org/cppdocs/installing.html#minimal-example}{Installing C++ Distributions of PyTorch};
        \item \href{https://docs.pytorch.org/cppdocs/notes/tensor_basics.html}{Tensor Basics};
        \item \href{https://docs.pytorch.org/docs/stable/generated/torch.nn.functional.conv2d.html}{torch.nn.functional.conv2d}.
    \end{itemize}
\end{itemize}

\end{document}
