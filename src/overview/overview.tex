\listfiles
\documentclass[12pt]{article}

\usepackage[T2A]{fontenc}
\usepackage[utf8]{inputenc}
\usepackage{indentfirst}

\usepackage[english, russian]{babel}
\usepackage[a4paper,top=2cm,bottom=2cm,left=3cm,right=3cm,marginparwidth=1.75cm]{geometry}
\usepackage[style=russian]{csquotes}
\usepackage{microtype}
\usepackage{hyperref}
\usepackage{caption}
\usepackage{graphicx}

\emergencystretch=2em

\title{Курс <<Гибридные вычислительные системы>>}
\author{ст. преп. каф. ВпВ ИКИТ СФУ Тарасов С. А.}
\date{}

\begin{document}
\maketitle

\abstract{}
Курс <<Гибридные вычислительные системы>> посвящен гетерогенным вычислениям (Heterogeneous Computing): рассматриваются общие аспекты гетерогенных вычислений; особое внимание уделяется технологии CUDA (GPGPU).
Продолжительность --- 2 семестра; формы оценивания: зачет, экзамен.

\section*{Гетерогенные вычисления}

Гетерогенные вычисления — это парадигма вычислений, в рамках которой для решения задач используются процессоры разных архитектур, каждый из которых оптимизирован для определённых типов операций. В отличие от гомогенных систем (например, только CPU), гетерогенные системы комбинируют разнородные вычислительные устройства в рамках одной платформы. Классическим примером HC-системы является связка CPU + GPU.

\section*{Содержание курса}
\begin{itemize}
\item Гетерогенные вычисления (обзор)
\item Обзор архитектуры NVIDIA Tesla GPU
\item Модель программирования CUDA
\item CUDA Toolchain: компиляция, линковка и дистрибуция
\item Тестирование
\item Точность вычислений и вычисления низкой точности
\item Проектирование гетерогенных вычислений/паттерны проектирования
\item Оценка производительности и оптимизация CUDA kernels
\item Математика глубокого обучения: автоматическое дифференцирование и обратное распространение ошибки
\item Реализация моделей глубокого обучения на базе CUDA
\end{itemize}

\section*{Практика}

Практическая часть курса состоит из 10 работ, которые охватывают основные составляющие разработки на CUDA; вот некоторые из них:

\begin{itemize}
\item написание и интеграция GPU-кода (kernels)
\item параллельное программирование и конфигурация сетки нитей
\item использование различных типов памяти (shared, global, texture и др.)
\item программирование тензорных ядер
\item внутриварповый обмен
\item работа с CUDA Streams \& Graphs
\item оптимизация kernels (register pressure, warp divergence, occupancy и др.)
\end{itemize}

Каждая работа имеет базовую структуру:

\begin{itemize}
\item разработка классов и kernels
\item интеграция kernels в C++-код
\item написание тестов и тестирование
\item написание бенчмарков и оценка производительности
\item анализ полученных оценок производительности
\end{itemize}

Все работы объединены тематикой глубокого обучения.
В результате успешного прохождения практической части курса студент получит навык реализации моделей глубокого обучения в фреймворке CUDA + PyTorch.

Задание каждой работы составлено таким образом, чтобы в результате ее выполнения студент получил программную библиотеку, которая будет использоваться в следующих работах.

К каждой работе прилагаются критерии ее оценивания.

Ознакомиться с заданиями можно по ссылке: \url{https://github.com/jzcurious/hsys-tasks.git}.

\section*{Технологический стек}
\begin{itemize}
\item Аппаратное обеспечение: NVIDIA Tesla GPU (Tesla T4)
\item CUDA Toolkit: NVCC, Nsight Compute, cuda-gdb, cuobjdump, Compute Sanitizer
\item Языки программирования: CUDA C++, C++ (std=C++20), NVIDIA PTX, Python
\item Библиотеки: CUDA Runtime API, CUDNN, Google Benchmark, Google Test, unittest/PyTest, Eigen, LibTorch, PyTorch, PyBind11
\item Сборка: CMake + Ninja
\item Компиляторы: NVCC, Clang++/G++
\item Git/GitHub
\item Google Colab
\item IDE: clangd, clang-tidy, clang-format
\item Паттерны проектирования: RAII, Data + View, Policy/Strategy, Expression Templates, CRTP и др.
\end{itemize}

\end{document}
